\documentclass[12pt]{article}

\usepackage{fancyhdr}   
\usepackage[utf8]{inputenc} % allow utf-8 input
\usepackage[T1]{fontenc}    % use 8-bit T1 fonts
\usepackage[colorlinks=true, linkcolor=black, citecolor=blue, urlcolor=blue]{hyperref}       % hyperlinks
\usepackage{url}            % simple URL typesetting
\usepackage{booktabs}       % professional-quality tables
\usepackage{amsfonts}       % blackboard math symbols
\usepackage{nicefrac}       % compact symbols for 1/2, etc.
\usepackage{microtype}      % microtypography
\usepackage{lipsum}		% Can be removed after putting your text content
\usepackage{graphicx}
\usepackage{natbib}
\usepackage{doi}
\usepackage{listings}
\usepackage{xcolor}
\usepackage{float}
\setcitestyle{aysep={,}}


% Header information
\newcommand{\Group}{Group Uno}
\newcommand{\Mike}{Mike Orduna}
\newcommand{\Noah}{Noah Murphy}
\newcommand{\Alex}{Alex Livingston}
\newcommand{\Class}{CS 3339: Computer Architecture}

\newcommand{\ReportPart}{Part one}
\title{Project Report: \ReportPart}

\author{
    \textbf{\Class} \\
    \textbf{\Group:} 
    \\
    \Mike \\
    \Noah \\
    \Alex \\
}
\date{\today}

\definecolor{codegreen}{rgb}{0,0.6,0}
\definecolor{codegray}{rgb}{0.5,0.5,0.5}
\definecolor{codepurple}{rgb}{0.58,0,0.82}
\definecolor{backcolour}{rgb}{0.95,0.95,0.92}

\lstdefinestyle{mystyle}{
    backgroundcolor=\color{backcolour},   
    commentstyle=\color{codegreen},
    keywordstyle=\color{magenta},
    numberstyle=\tiny\color{codegray},
    stringstyle=\color{codepurple},
    basicstyle=\ttfamily\footnotesize,
    breakatwhitespace=false,         
    breaklines=true,                 
    captionpos=b,                    
    keepspaces=true,                 
    numbers=left,                    
    numbersep=5pt,                  
    showspaces=false,                
    showstringspaces=false,
    showtabs=false,                  
    tabsize=2
}

\lstset{style=mystyle}

\begin{document}
\maketitle

\newpage
\tableofcontents
\thispagestyle{empty}
\pagestyle{fancy}

\newpage
\setcounter{page}{1}
\section{Introduction}
The goal of this project is to explore the process of designing a computer circuit from the ground up, focusing on the micron level. 
Specifically, we will develop an integer Arithmetic Logic Unit capable of performing fundamental mathematical operations, including 
addition, subtraction, multiplication, and division. To achieve this, we began by constructing basic logic gates such as NOT, AND, and OR. 
These one-bit gates serve as the foundation for the larger circuit, enabling a modular and hierarchical design approach. Each module builds 
upon these basic components, ensuring scalability and reusability throughout the development process. Once the design phase was complete, 
rigorous testing was conducted to validate the functionality of the ALU. A dedicated testbench was created to simulate various scenarios and 
verify the expected behavior of the circuit. The results of these simulations were visualized using the GTKWaveform tool, allowing for detailed 
analysis and comparison of the actual outputs with the expected outcomes. After confirming that the test results matched our expectations, 
we successfully completed the first phase of the project and are now prepared to move forward with its next stages.

\section{Design}
Before designing the ALU itself, we first established a methodological approach to its development, breaking the project into manageable 
sections that build upon one another. This resulted in a hierarchical design, where lower-level modules serve as the foundational building 
blocks for higher-level components. Our first step was to design the basic logic gates: NOT, AND, OR, NAND, NOR, XOR, and XNOR. These gates 
are essential to the ALU's operation and were implemented using Boolean algebra principles, ensuring reusability and consistency throughout 
the design.

\subsection{Basic Logic Gates}
Each gate plays a pivotal role in enabling the ALU to perform its operations: the AND and OR gates facilitate logical decision-making, the NOT 
gate allows for signal inversion, and the XOR gate is critical for arithmetic functions like addition. More complex gates, such as NAND and XNOR, 
were constructed by instantiating simpler gates, aligning with the modular approach we outlined. This design choice streamlined the debugging 
process and provided a scalable framework for representing the ALU. These gates formed the first steps in creating a robust and extensible model 
that would support more complex operations in subsequent stages of the project.

\section{Verilog Code}
Descriptions of the Verilog code

\subsection{Implementation}
Reasons for the implementation

\subsection{Testbench Development}
Methods of testing and validation

\section{Waveform Analysis}
Analysis of the simulation results

\subsection{GTKWave Overview}
Overview of the GTKWave tool

\subsection{Simulation Results}
Results of the simulation tests

\section{Discussion}
Discussion of the project's expectations, outcomes, and potential improvements

\section{Conclusion}
Summary of the project's objectives and achievements

\end{document}

